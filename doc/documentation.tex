%%%%%%%%%%%%%%%%%%%%%%%%%%%%%%%%%%%%%%%%%
% University/School Laboratory Report
% LaTeX Template
% Version 3.0 (4/2/13)
%
% This template has been downloaded from:
% http://www.LaTeXTemplates.com
%
% Original author:
% Linux and Unix Users Group at Virginia Tech Wiki 
% (https://vtluug.org/wiki/Example_LaTeX_chem_lab_report)
%
% License:
% CC BY-NC-SA 3.0 (http://creativecommons.org/licenses/by-nc-sa/3.0/)
%
%%%%%%%%%%%%%%%%%%%%%%%%%%%%%%%%%%%%%%%%%

%----------------------------------------------------------------------------------------
%	PACKAGES AND DOCUMENT CONFIGURATIONS
%----------------------------------------------------------------------------------------

\documentclass{article}
\usepackage[utf8]{inputenc}
\usepackage{polski}
\usepackage[polish]{babel}
\usepackage{cite}
\usepackage{url}


\usepackage{graphicx} % Required for the inclusion of images

\setlength\parindent{0pt} % Removes all indentation from paragraphs

%\renewcommand{\labelenumi}{\alph{enumi}.} % Make numbering in the enumerate environment by letter rather than number (e.g. section 6)

%\usepackage{times} % Uncomment to use the Times New Roman font

%----------------------------------------------------------------------------------------
%	DOCUMENT INFORMATION
%----------------------------------------------------------------------------------------

\title{Modelowanie zawodów Formuły 1} % Title

\author{Sabina \textsc{Rydzek} \\ Mateusz \textsc{Kotlarz} \\ Kacper \textsc{Furmański}} % Author name

\date{\today} % Date for the report

\begin{document}

\maketitle % Insert the title, author and date


\begin{abstract}
Projekt ten obejmuje stworzenie modelu wyścigu Formuły 1 oraz jego symulację na domyślnie wczytanym torze Hungaroring Circuit. Wykorzystuje automaty komórkowe. Symulowane są różne czynniki i typy nawierzchni mające wpływ na trasę i sposób jazdy bolidów. Dodane są również czynniki atmosferyczne. Umożliwia odczyt prędkości poszczególnych bolidów. Projekt jest otwarty i umożliwia tworzenie i wczytywanie innych tras w odpowiednim formacie.
\end{abstract}

%----------------------------------------------------------------------------------------
%SECTION 1
%----------------------------------------------------------------------------------------

\section{Automaty komórkowe}

W projekcie zastosowany jest model automatów komórkowe. Jest to dyskretny, nieliniowy model, który składa się ze skończonej oraz uporządkowanej liczby \textit{komórek}, które w każdej chwili posiadają swój z góry określony stan, który wyznaczany jest tylko na podstawie komórek sąsiadujących (wg. wybranego sąsiedztwa) \cite{cellularAutomata}. \\

Takie podejście pozwala wyeliminować nadmiarowe obliczenia oraz potrzebę śledzenia pozycji. Dodatkowo, pozwala na dodatkowe optymalizacje przy wykorzystaniu programowania równoległego. \\

Tor F1 oraz jego otoczenie jest tablicą komórek o różnych współczynnikach w zależności od ich typu (trawa, krawężnik, asfalt itp.). Głównym problemem w zastosowaniu modelu automatów komórkowych jest to, że bolid jako ciało stałe zajmuje więcej niż jedną komórkę. Oddziaływanie na siebie wielu \textit{bloków} komórek kłóci się z podstawowym założeniem, że w obliczeniach pod uwagę brane są tylko komórki w określonym sąsiedztwie \cite{particleSimulation}. Aby poradzić sobie z tym problemem, bolid przedstawiony jest tylko jako środek ciężkości (jego obrzeża są nakładką na torze), a sąsiedztwo zostało odpowiednio rozszerzone.

\subsection{Grid} 

Wybraliśmy Hungaroring Circut, znajdujący się w pobliżu Budapesztu. Ma on $4.381$km długości oraz $14$ zakrętów \cite{track}. 

\subsection{Czas}

Automaty komórkowe nie działają w \textit{czasie rzeczywistym}. Zmiana czasu to stworzenie nowej generacji komórek na podstawie określonych reguł ich zachowania \cite{nature}. 

\subsection{Typy komórek}
\begin{itemize}
\item SurfaceType - Road, Grass, Worse Road, Sand, Barrier, None - określają typ nawierzchni oraz przypisane im współczynniki, które wpływają na m.in. na szybkość, przyczepność
\item Direction -  Top-left, Top, Top-right, Left, Right, Bottom-Left, Bottom, Bottom-Right - określają prawdopodobny kierunek poruszania się bolidu po torze, wykorzystywane przy określaniu idealnej trasy
\item Angle -  wykorzystywany przy modelowaniu zakrętów
\end{itemize}

\subsection{Stany}

Komórka może mieć stan \textit{zajęty}, kiedy znajduje się tam środek ciężkości bolidu, oraz \textit{wolny}. Sam bolid jest jedynie nakładką na tor.

\subsection{Reguły}
todo
%----------------------------------------------------------------------------------------
\bibliographystyle{plain}
\bibliography{bibliography}

\end{document}